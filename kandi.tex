% vim:tw=0
\documentclass[a4paper,finnish,12pt]{scrartcl}
\usepackage{mythesis}
%
\author{Sakari Cajanus}

\title{Läheisyyden ja sukulaisuuden vaikutus ihmisten välisen kommunikaation klusteroitumiseen}{Och samma på English}
\place{Espoo}
\thesisdegree{Kandidaatintyö}{Bachelor's Thesis}
\instructor{FT Mari Myllymäki}{Ph.D. Mari Myllymäki}
\supervisor{TkT Markus Turunen}{D.Sc. (Tech.) Markus Turunen}

\uni{Aalto-yliopisto}{Aalto University}
\school{sähkötekniikan korkeakoulu}{School of Electrical Engineering}
\degreeprogram{Bioinformaatioteknologia}{Bioinformation technology}
\department{Lääketieteellisen tekniikan ja laskennallisen tieteen laitos}{Department of Biomedical Engineering and Excellence in Computational Science}
\keywords{suomeksi}{englanniksi}

\hypersetup{%
    colorlinks=true, linktocpage=true, pdfstartpage=3, pdfstartview=FitV,%
    % uncomment the following line if you want to have black links (e.g., for printing)
    %colorlinks=false, linktocpage=false, pdfborder={0 0 0}, pdfstartpage=3, pdfstartview=FitV,% 
    breaklinks=true, pdfpagemode=UseNone, pageanchor=true, pdfpagemode=UseOutlines,%
    plainpages=false, bookmarksnumbered, bookmarksopen=true, bookmarksopenlevel=1,%
    hypertexnames=true, pdfhighlight=/O,%hyperfootnotes=true,%nesting=true,%frenchlinks,%
    urlcolor=blue, linkcolor=blue, citecolor=green, %pagecolor=RoyalBlue,%
    %urlcolor=Black, linkcolor=Black, citecolor=Black, %pagecolor=Black,%
    pdftitle={\thetitle},%the title
    pdfauthor={\theauthor},%your name
    pdfsubject={},%
    pdfkeywords={\thekeywords},%
    pdfcreator={XeLaTeX},%
    pdfproducer={A happy XeLaTeX user}%
}


\begin{document}

\maketitlepage
\newpage

\begin{abstractpage}
Kuopion rannassa soivat lähtevien laivain kellot jo toiseen kertaan.
»Elias Lönnrot» oli isoin ja komein laiva, joka vei matkustajia etelään
päin, Helsinkiin ja muuanne, ja sen kello aloitti ensiksi. Sen soidessa
yhtyivät soimaan pienempäin laivain kellot, ja ne helistivät yht'aikaa,
kilvan keskenään.
\end{abstractpage}
\begin{abstractpage@en}
Kuopion rannassa soivat lähtevien laivain kellot jo toiseen kertaan.
»Elias Lönnrot» oli isoin ja komein laiva, joka vei matkustajia etelään
päin, Helsinkiin ja muuanne, ja sen kello aloitti ensiksi. Sen soidessa
yhtyivät soimaan pienempäin laivain kellot, ja ne helistivät yht'aikaa,
kilvan keskenään.
\end{abstractpage@en}
%\auth
%\authorname
\section{Juttu!}
ff
ri
fi
Kuopion rannassa soivat lähtevien laivain kellot jo toiseen kertaan.
»Elias Lönnrot» oli isoin ja komein laiva, joka vei matkustajia etelään
päin, Helsinkiin ja muuanne, ja sen kello aloitti ensiksi. Sen soidessa
yhtyivät soimaan pienempäin laivain kellot, ja ne helistivät yht'aikaa,
kilvan keskenään.

\begin{verbatim}
Oli sunnuntai-aamu, jolloin joutilas väki vartavasten tulee rantaan
katsomaan, ketä lähtee. He tulevat kiireissään yli satamatorin, ovat
hommassa, niinkuin olisi heillä pelko myöhästymisestä. Laivasillalle
tultua katsellaan tuikeasti ja terävästi, niinkuin olisi asia jotain
tavata. Mutta sitten rauhoitutaan ja asetutaan vähän taammaksi
tarkastamaan niitä, jotka ovat lähtevän näköisiä.
\end{verbatim}

Antti oli ylioppilas. Hän oli suorittanut tutkintonsa mennä kevännä ja
oli nyt lähdössä ensi lukukaudekseen Helsinkiin lukemaan. Hän seisoi
laivasillalla lähellä kulkulautaa. Päässä oli valkoinen rutistumaton,
sisarien puhtaaksi pesemä ylioppilaslakki ja kaulassa riippui nahkainen
matkalaukku, hiukan takana päin vasemman lanteen käänteessä. Palttoo
oli nähtävästi uusi, eilen räätälistä saatu, samettikauluksinen, ja
pienestä rintataskusta piipotti punaraitainen nenäliina, joka juuri
kotoa lähdettäessä oli siihen siististi ja varovasti hypistetty.

--Kuule, Antti, eiköhän sinun jo pitäisi nousta laivaan ... se saattaa
lähteä, ja sinä ehkä jäät ... jos ne ottavat pois laudan...

Se oli hänen äitinsä, joka häntä laivaan kiirehti. Koko perhe oli
tietysti saattamassa, nimittäin isä, äiti, kaksi sisarta ja muutamia
sukulaisia ja tuttavia. He seisoivat ahtaassa ryhmässä Antin ympärillä
ja odottivat vain kellon kolmatta kalahdusta, joka aina tuntui viipyvän
niin kauhean kauan.

Antti oli nähtävästi pahalla tuulella. Otsa rypyssä ja olkapäitään
hermostuneesti kohotellen seisoi hän äitinsä vieressä syrjittäin
häneen. Äiti oli pieni, matala ja hento eikä ylettynyt poikaansa kuin
parahiksi kainaloon. Levottomana ja huolekkaana koetti hän kuitenkin
saavuttaa häntä edes silmiin näkemään. Mutta ne harhailivat sinne tänne
yli väkijoukon, eikä äiti tahtonut saada pojan huomiota puoleensa,
jonka olisi viimeisellä eron hetkellä halunnut kokonaan omistaa.

--Ketä sinä, Antti, katselet? Onko sinulla siellä joku, jota tahdot
tavata?

--Ei ole.

Antti käänsi kasvonsa toisaanne päin haihduttaakseen noita, jotka
seurasivat jokaista hänen liikettään ja kasvojensa värettä. Mutta vähän
ajan päästä kiintyi hänen huomionsa kuitenkin taas sinne, missä oli
väkeä kokoontunut toiselle laivasillalle »Ilman» lähtöön.

--Muistitko, Antti, ottaa kalvokkaat käteesi?--Eihän sinulla niitä
olekaan!

--Kuinka minä niitä nyt käteeni?

--Minkä tähden et ... järvellä tuulee niin kylmästi avonaiseen
hihaan ... vaan ehkä sinä oletkin salongissa ... olekin vaan
salongissa, Antti!

--Kyllä, kyllä!

--Miksi sinä, Antti, olet niin kärsimätön? sanoi äiti vähän
loukkaantuneena.

--Mutta minkä tähden et sinäkään anna sen olla rauhassa? murahti siihen
isä, joka seisoi toisella puolen Anttia ja oli tyytymätön siitä, ettei
odotus loppunut.--Tottahan tuollainen iso poika jo katsoo itsensä ...
etkähän sinä sen perässä enää kuitenkaan saata sen pitemmälle hypätä.

Äidin leuka alkoi yhä tiheämmin vavahdella, ja hän koki turhaan niellä
itkuaan. Sisarienkin silmät rupesivat kostumaan, eikä nenäliinoja enää
pantu taskuun. Anttia tämä hellyys hävetti, eikä hän tuntenut
vähintäkään sääliä äitiänsä kohtaan, pikemmin harmia. Hän olisi jo
mennyt laivaan, ellei äiti olisi tarttunut hihaan ja pidellyt sitä kuin
pihtien välissä.

Ei kukaan puhunut pitkään aikaan. Ympärillä vain kuhisi ja kihisi
ihmisiä, jotka yhtämittaa survivat milloin selkään milloin kylkeen.
Joutilaita kokoontui yhä enemmän, ne jäivät seisomaan siihen
kerroksittain tukkien tien toisilta, joilla oli asiaa mennä ja tulla.
Ajurien kärryt kolisivat, hevosten kaviot kalkkoivat kiviseen
tantereeseen, ja yhä uusia ajoi kaupungista. Kapsäkeillä, kirstuilla,
kohverteilla ja kaikenlaisilla kapineilla sälytetyitä kärryjä
peruutettiin aivan laivasillan reunalle, ja siitä tavarat mätettiin
laivaan. Kapteeni hääri edestakaisin ja huusi ja komensi heikonlaisella
suomenkielellä. Ja laiva puhkui malttamattomuudesta. Piipuista ja
kupeessaan olevasta reiästä puhalsi se valkoista höyryä, joka peitti
peräpuolen sakeaan pilveen. Konekin jo hiljaa käydä jyskytteli,
koetteli voimiaan, ja köydet olivat pingollaan kuin juoksuhaluisen
syöttilään ohjakset, jota ajaja ei päästä menemään, mutta joka
hirnahtelee ja kuolaimiaan pureskellen jalkojaan nostelee.

Sitä katselivat siinä Antti, hänen isänsä ja sisaret, ajattelematta,
mitä katselivat. Mutta äiti vain katseli Anttia.

Isä oli pienenlainen, vähän kumarahartiainen mies, ja hänen palttoonsa
kaulus oli pystyssä. Kädessään oli hänellä päästään koukistettu meren
ruoko. Sisaria oli kaksi, jotenkin yhdennäköisiä. Molemmat he olivat
puetut aivan samanlaisiin vaatteisiin. Vanhempi oli vähän vakavamman ja
kuivemman näköinen. Nuoremmassa sitä vastoin näkyi vielä tuota
notkeutta ja mehua, joka on ominaista kaksikymmenvuotiaalle, mutta joka
pikkukaupungeissa tavallisesti niin pian haihtuu.

Yhdessä kohden seisominen kävi hänelle ikäväksi ja hän kääntyi
puhelemaan erään »hyvän ystävänsä» kanssa.

--Sinun veljesi menee Helsinkiin? kuuli Antti ystävän sopottavan.

--Niin hän menee.

--Tuleeko hän jouluksi kotiin?

--Kyllä hän tulee.

--No, onko sinulla ikävä häntä?

--Kyllä tuntuu tyhjältä, kun hän on poissa.

--Eikös sinulla ole muitakin ikävä...?

--Ketä muita?

--Pekkahan menee kanssa tässä laivassa ... missä hän on, kun ei häntä
vielä näy?

--Voi, voi, en minä tiedä, ja laiva kohta lähtee!

--Tuolla hän jo juoksee!

Pekka, joka sieltä tuli, oli näet Antin nuoremman sisaren sulhanen.
Eivät he kuitenkaan vielä julkikihloissa olleet. Ainoastaan
sukulaisille oli sanottu asiasta, jota ei tahdottu ilmaista ennen kuin
Pekka oli suorittanut papintutkintonsa, joka tapahtui jouluksi. Oli
Anna kertonut siitä myöskin joillekuille parhaille ystävilleen suurena
salaisuutena. Ja muuten tiesi sen sitä paitsi koko kaupunki.

Yhtäkaikki tervehti Pekka yhtä kylmästi ja kohteliaasti morsiantaan
kuin kaikkia muitakin, sillä hän osasi käyttäytyä ja taisi hillitä
tunteensa. Hän tuli kapsäkkiään kantaen, rynnisti leveillä hartioillaan
seisovia syrjään ja teki tervehdyksensä itsetietoisen tyytyväisesti.

Kun hän oli kaikkia tervehtinyt ja antanut kättä kaikille, ryhtyi hän
vilkkaasti puhelemaan ja selittämään, minkä tähden oli viipynyt. Ei
ollut saanut ajuria j.n.e. Suurella uteliaisuudella sitä sisaret
kuuntelivat ja kaikki sukulaisystävät, sillä Pekka oli aina ollut
tätien lemmikki. Hän oli siivo poika, kuului raittiusseuraan, ei
tupakoinut, joi ainoastaan kahvia, teetä ja vesiä, luki ahkeraan ja
jutteli kovin mielellään.

Vaan nyt keskeytti hänet Antin äiti. Hän pyysi Pekkaa, joka oli
vanhempi ja kokeneempi, pitämään huolta Antista ja neuvomaan häntä kuin
nuorempaa veljeään. Ja Antin piti kuulla ja ottaa vaaria siitä, mitä
Pekka hänelle sanoi.

Antti kohotti kärsimättömästi olkapäitään, sillä hän ei voinut sietää
Pekkaa. Sellainen sileänaamainen esimerkki, josta hänen alinomaa
käskettiin vaaria ottamaan. Hän ei suinkaan ollut mikään lapsi enää,
joka ei olisi kyennyt toisten taluttamatta kävelemään. Hän oli
täysikasvuinen mies ja kokenut enemmän kuin moni muu nuori mies hänen
iällään. Johan hän oli ollut onnettomasti rakastunutkin, ja oli
parhaallaan.

Pekka vastasi äidin kehoitukseen ainoastaan epämääräisellä hymyilyllä,
sillä sisaren sulhasena hän tahtoi olla hyvissä väleissä veljen kanssa,
jota kaikki kotona ihailivat.

--Minä saan sanoa terveisiä sisareltani ... minä juuri saatoin hänet
»Ilmaan» ... hän olisi tullut hyvästille tänne, mutta pelkäsi, että jos
laiva jättää.

Molemmat sisaret tietysti voivottelemaan sitä, kun eivät saaneet sanoa
Almalle hyvästiä. Jos ehtisi vielä mennä! Mutta ei nyt enää ehdi.

Kun Antti kuuli Almasta puhuttavan, tuli hänen katseensa epävarmaksi,
ja hän tunsi punastuvansa.

Pekalla oli kapsäkkinsä koko ajan kädessään, ja sisaret kehoittivat
häntä viemään sen laivaan ja tulemaan sitten takaisin vielä
juttelemaan.

Mutta silloin katsoi isä kelloaan ja ilmoitti miltei vihaisesti, ettei
ole aikaa jälellä kuin puolitoista minuuttia. Joka menee, niin sen on
parasta heittää hyvästinsä ja jäädä laivaan samalla.

Äiti tarttui Anttia napinläpeen ja veti hänet muista vähän erilleen.

--Pane sinä palttoosi kiinni, Antti ... nyt tuulee niin kylmästi... Jos
laivassa nouset kannelle, niin ota turkki hartioillesi. Mutta missä on
sinun kaulaliinasi?

--Siellä se on jossain ... en minä tiedä.

--Ja kirjoita sinä ainakin kerta viikossa, jos et joka postissa ennätä.

--Kyllä minä kirjoitan.

--Ja koeta nyt, Antti, elää niin säästäväisesti kuin mahdollista, kun
pappa tahtoo niin tarkkaa tiliä... Sinun on ehkä vaikea panna kirjaan
kaikkia pikkumenoja, niin tässä on sinulle vähän käsirahaa, joita minä
olen säästänyt ... jos sattuu mielesi tekemään leivoksia tai jos
tarvitset jonkun rusetin tai jonkun...

Mutta silloin lyödä kalahti kieli kellon laitaan kolmannen kerran, ja
kiireesti sai äiti pistetyksi paperikäärön Antin käteen.

Sitten sulki hän Antin hellään syliinsä. Ripustautui hänen kaulaansa ja
puristi, tahtomatta koskaan heittää. Hän olisi melkein tahtonut pitää
häntä kiinni niin kauan, että laiva olisi ehtinyt jättää... Nyt hän
menee, lähtee ulos suureen maailmaan, kuka tiesi minkälaisia kohtaloita
kokemaan! Ja äiti tavoitti suutelemaan poikaansa, kohosi varpailleen ja
koki kurkottautua hänen huulilleen. Mutta poika vältti. Hän piti
itseään jo liian suurena suudellakseen äitiään näin kaikkien nähden. Ja
pudisti sentähden vain hätäisesti sekä hänen kättään että muiden käsiä
ja kiiruhti huojuvaa kulkulautaa myöten laivaan.

Pekka heitti hellemmät jäähyväiset heille kaikille, kiitti ja kumarsi
matkalle toivotetusta onnesta ja koki pusertaa harmaista,
matalapohjaisista silmistään syviä silmäyksiä jokaiseen, jonka kättä
hän puristi.

--Otta landgongi pois! huusi kapteeni, ja se kiskaistiin laivasillalle
melkein Pekan jalkain alta.

Antti nousi komentosillalle ja koetti seisoa siellä niin, että hänen
asentonsa näyttäisi siltä, kuin hän olisi aikuinen mies. Äidin
liiallinen hellyys, isän vakavuus ja varmuus ja sisarien suojelevat
silmäykset, ne olivat tähän saakka estäneet hänen rinnassaan asuvia
aikamiehen tunteita oikein vaikuttamasta. Mutta nyt seisoivat holhoojat
tuolla alhaalla pieninä ja voimattomina. Hän seisoi täällä, ylempänä
heitä, lähtemässä, itsenäisen miehen merkki otsassa. Vasen käsi oli
taskussa ja oikea poveen pistettynä. Nenäliinan nurkka yhä piipotti
rintataskusta esille.

Laivaa siirrettiin ulommas, ja juopa laidan ja laivasillan välillä
suureni yhä.

Alhaalta kuului vielä ääni huutavan:

--Antti, muista nyt, mitä olen sanonut!--Hyvästi Antti! Hyvästi,
Antti!--jota seurasi melkein epätoivoisen kiihkeä nenäliinan huiskutus.

Pekka raivosi vastaan omalla nenäliinallaan, juoksi kantta pitkin,
nosteli lakkiaan ja nyökytteli päätään. Antti ei tehnyt yhtään mitään.
Hän pysyi liikkumatonna perämiehen vieressä ja koskettihan vain pari
kertaa lakkinsa lippaa.

--Huiskuta sinäkin, Antti, etkö näe, kuinka kaikki muut huiskuttavat!

Silloin täytyi Antinkin vetää nenäliinansa esille. Vitkaan kohotti hän
sitä pari kolme kertaa ja pisti sitten taas taskuunsa.

»Ilma» oli lähtenyt laituristaan samaan aikaan ja mennä suhkasi jo
täyttä vauhtia suuremman ja hitaamman laivan keulan editse.

--Hyvästi Pekka ja Antti! huudahti yht'äkkiä sulava naisääni »Ilmasta».

Alma, Pekan sisar, seisoi siellä solakkana »Ilman» perässä, piteli
viiritangosta ja liehutti hänkin toisella kädellä valkoista liinaa.
Pekka alkoi heti kohta riehua sinnepäin; mutta Antti koetti kohottaa
lakkiaan niin kohteliaasti kuin suinkin.

Alma oli se, johon hän oli ollut onnettomasti rakastunut ja oli
vieläkin. Se oli ikävä, surullinen juttu, siksi oli hän nyt jäykkä
kaikille, siksi hän ei voinut olla hellä omaisilleenkaan, sentähden
halusi hän päästä vapauteen, ja sentähden »halveksi hän koko maailmaa.»
Nyt poistui hän tuonne pohjoiseen, Antti lähti etelää kohti. Heidän
tiensä erosivat, erosivat ikuisesti. Ja kun valkoinen höyrypilvi
yht'äkkiä laski »Ilman» piipusta ja peitti vaippaansa koko peräpuolen
laivaa ja Almakin sinne katosi, niin olisi Antin tehnyt mieli levittää
kätensä ja huudahtaa katkera »haa!» Mutta kun kapteeni samassa komensi
koneen täyteen työhön ja kun propelli perän alla särpäisi vettä
rintansa täydeltä ja laiva nytkähtäen ojensihe menemään, niin vaihtui
mieliala samassa niinkuin sen, jota yht'äkkiä pudistetaan hereille
pahasta painajaisesta. Antin rinnassa hyppäsi riemun tunne päällepäin,
riippumattomuus riipaisi mukaansa ja tuntui siltä, kuin elämä nyt vasta
olisi oikein irtautunut luistamaan.

--Nyt se meni, huokaisi äiti rannalla, josta ei lähtenyt, ennenkuin
laiva katosi Väinölänniemen taa. Valkoista höyryä näkyi vielä hetkinen
puiden yli, ja sitä hän astuessaan rantatoria piti vielä silmällä,
kunnes sekin katosi.

Äänetönnä poistui perhe rannasta. Isä astui edellä, kalkutellen
rautapääkepillään katukiviin. Vähän jälempänä kulki nuorempi sisar,
mutta vanhempi odotti äitiä ja käveli hänen kanssaan rinnan. Ei kukaan
virkkanut sanaakaan. Isä ei tavallisesti juuri milloinkaan puhunut,
nuoremmalla sisarella oli omat syynsä äänettömyyteen, ja vanhempi sisar
oli vaiti siksi, että äitikin oli vaiti.

Sillä tavalla palasivat he asuntoonsa. Pääovi oli kiinni, ja isä ja
Anna jäivät odottamaan, että tultaisiin avaamaan. Odottaessaan piirteli
Anna hajamielisesti kuvioita pihahiekkaan päivävarjonsa kärellä, ja
isä, joka oli hyvin järjestystä rakastava, määräsi, että havut rappujen
edessä ovat uudistettavat. Äiti meni vanhemman sisaren kanssa keittiön
\end{document}
